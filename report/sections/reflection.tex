
\section{Reflection}
\label{sec:Reflection}

This project began as an exploration of applying tools from programming language
theory to hardware architecture.  The notion of a hash function for homing
memory to tiles is a well-defined component of a processor and has the advantage
of being relatively abstract, without requiring large amounts of
hardware-specific knowledge.  While we ultimately derived two interesting
strategies for solving this problem, one of which was indeed based on a tool
from programming language theory, the progress we made was built on a couple of
weeks of background reading as well as a fair amount of trial and error.

We began by reading about caches and homing strategies in manycore designs until
we felt we had a pretty good idea of what the bounds of the problem were.  We
also spent some time reading about hash functions, learning about their
typology, including the universal, cryptographic, and hardware varieties.  On
the programming language front, we decided to explore a synthesis methodology,
and worked to build a few examples of simple synthesis tasks over the theory of
bitvectors.  Throughout our investigations, we also learned about how problems
similar to this were addressed in the machine learning space.

Our first few attempts at approaching the problem stalled for a few reasons.
While the requirements of the problem are easy to state in prose, it was not
clear how to state this in our synthesis tool.  Namely, we want a hash function
that distributes memory addresses to tiles according to a specific distribution
in aggregate and is based on simple bitvector operations.  As we went
back-and-forth trying to express these requirements, we ended up tackling the
problem from the analytic approach as well as the synthesis.  It turned out that
thinking about the problem from both ends is how we eventually were able to
provide \textit{two} distinct solutions to the problem.  The code, sweat, and tears are
all included in the commits of this repo, including programs in both Racket and
Python that may be explored by the interested reader.


