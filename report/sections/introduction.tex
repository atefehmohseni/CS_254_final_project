\section{Introduction}
\label{sec:Introduction}

In tiled many-core processors design processing cores are fitted onto a single
chip and cores are interconnected via mesh-based networks. There are local and
remote memory controllers for any tile (processing core) that are fitted onto
the same chip \cite{tiled-manycore}. When a core wants to load or store to a
memory address, it needs to map the cache line address to find the corresponding
tile that it's sharing status is stored. In this report we will explore
different mapping algorithms to efficiently map memory addresses to the caches
given we have different tile sizes.

We explored two main approaches to solving this problem, program synthesis and
parameterized hashing.  The first approach is based on a technique used widely
in the field of programming language theory.  In program synthesis, we state a
problem based on a specification of its constraints and allow a domain specific
engine or logical solver to explore the problem space for satisfying solutions.
In parametrized hashing, we investigate how to make existing hash functions
sensitive to parameters describing the bins, or output, of the function,
ultimately deciding on a scheme based on voting theory. Finally, we canvas
alternate places where we might find solutions to this problem, specifically in
the field of machine learning.
